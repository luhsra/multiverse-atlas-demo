%*****************************************************************************
%** beamer setup  **************************************************************

\usecolortheme{rose}
%\usetheme[wide]{sra}
\usetheme[wide,bgstyle=osg]{sra}

\setbeamertemplate{navigation symbols}{}

\usepackage[sracolors,autonotes]{beamertools}

\usepackage{csquotes}
\usepackage{booktabs}
\usepackage{xspace}
\usepackage{multirow}
\usepackage{pgffor}

\usepackage{iftex}
\ifluatex
  \usepackage{lmodern}
  \usepackage[no-math]{fontspec}
  % VeraMono ist viel schmaller als Beramono
  \setmonofont{VeraMono}[
     Extension       = .ttf,
     BoldFont        = *-Bold,
     ItalicFont      = *-Italic,
     BoldItalicFont  = *-Bold-Italic,
     Scale           = MatchLowercase,
  ]
  \usepackage{pifont}
\else
  \usepackage[utf8]{inputenc}
  \usepackage[T1]{fontenc}
  \usepackage{helvet}
  \usepackage{pifont}
  \usepackage[scaled=0.85]{beramono}
\fi

\long\def\maketitleframe{
  \begin{frame}
    \maketitle
  \end{frame}
}


\newcommand{\dividerframe}[1]{
  \begin{frame}
    \begin{center}
      \Huge #1
    \end{center}
  \end{frame}
}

\AtBeginDocument{
  \pagenumbering{arabic} % for pageslts
}

% Customize numbering:
% We want to number frames as <chapter>-<frame> with <frame> being by-chapter numbers
% Note: The following works only if a patch is applied to beamerbaseframe.sty (see README)
\renewcommand{\theframenumber}{\arabic{framenumber}}
\renewcommand{\insertframenumber}{\theframenumber}
\renewcommand{\InsertFrameNumber}{\insertframenumber}

  \newcommand{\columntitle}[1]{
    {\centering\structure{\strut #1}\par}
  }


  \newcommand{\animation}[3][]{%
  \foreach \p/\s in {#2} {%
    \includegraphics<\s|handout:\s|skript:\s>[page=\p,#1]{#3}%
  }%
}

\usepackage{mdframed}


%*****************************************************************************
%** bibliography
%*************************************************************
\usepackage{polyglossia}
\setdefaultlanguage{german}

\usepackage[style=numeric-comp,hyperref,maxnames=3,minnames=3,defernums=true]{biblatex}
\defbibheading{bibliography}{}
\bibliography{bib/sra-own.bib}
\renewcommand{\bibfont}{\scriptsize}
\DeclareFieldFormat{postnote}{#1}



%****************************************************************
%** tikz stuff
\usepackage{tikz}
\usetikzlibrary{tikzmark,arrows,fit,calc,positioning,sra,shapes.multipart}
\usepackage{pgfkeys}
\tikzset{
  >=latex',
  small tree/.style={
    every node/.append style={
      font=\footnotesize,
      draw,
    },
    level distance=1cm,
  }
}

%*****************************************************************************
%** Legalcode *************************************************************

\pgfkeys{
  /legalcode/commons/.style={url={https://commons.wikimedia.org/wiki/File:#1}},
  /legalcode/url/.code={%
    \def\lcText##1{\href{#1}{##1}}
  },
  /legalcode/author/.code={\def\lcAuthor{, #1}},
  /legalcode/license/CC BY-SA 3.0/.code={%
    \def\lcLicense{\href{https://creativecommons.org/licenses/by-sa/3.0/legalcode}{CC BY-SA 3.0}}%
  },
  /legalcode/license/Free Art License/.code={%
    \def\lcLicense{Free Art License}%
  }
}
\newcommand{\legalcode}[3][]{% [opts]{license}{title}
  \bgroup%
  \def\lcText##1{##1}
  \def\lcAuthor{}
  \pgfkeys{/legalcode/.cd,#1,license/#2}%
  \parbox{\textwidth}{\centering\color{gray}\tiny\lcText{#3}\lcAuthor{} \mbox{(\lcLicense)}}
  \egroup
}

\colorlet{codecolor}{luhlightgray!50}
\usepackage{lecturesourcecode}




%*****************************************************************************
%** code blocks **************************************************************

  \makeatletter
  \setbeamercolor{code}{fg=black,bg=codecolor}

  \define@key{beamercolbox}{text width}{\pgfmathsetlengthmacro{\beamer@colbox@wd}{#1}}
  \define@key{beamercolbox}{text height}{\pgfmathsetlengthmacro{\beamer@colbox@ht}{#1}}
  \define@key{beamercolbox}{text depth}{\pgfmathsetlengthmacro{\beamer@colbox@dp}{#1}}
  \define@key{beamercolbox}{background color}{\setbeamercolor{code}{fg=black,bg=#1}}
  \define@key{beamercolbox}{scale content}{\def\bt@innerscale{#1}}
  \define@key{beamercolbox}{tag}{\def\bt@tag{#1}}

  \newenvironment<>{code}[1][\empty]{%
    \begin{onlyenv}#2%
    \vspace{2pt}%
    \setkeys{beamercolbox}{#1}
    \lstset{aboveskip=0pt,belowskip=0pt,linewidth=\beamer@colbox@wd}%
    \rule{2pt}{0pt}\begin{beamercolorbox}[colsep*=2pt,text width=\linewidth-4pt, text depth=0ex,#1]{code}%
    \ifdef\bt@innerscale{%
      \pgfmathsetlength{\@tempdima}{\beamer@colbox@wd/\bt@innerscale}
      \begin{adjustbox}{scale=\bt@innerscale}
      \begin{minipage}{\@tempdima}
        \lstset{linewidth=\@tempdima}%
      }{}
  }{%
    \ifcsdef{bt@tag}{%
      \vspace{-1.2\baselineskip}\hfill{\emph{\color{black!70!white}\csuse{bt@tag}}}
      \vspace{0.4\baselineskip}
    }{}%
    \ifdef\bt@innerscale{%
      \end{minipage}
      \end{adjustbox}
    }{}%
    \end{beamercolorbox}\rule{2pt}{0pt}\vspace{2pt}%
    \end{onlyenv}%
    \ignorespacesafterend%
  }

  \usepackage{xparse}
  \usepackage{realboxes}


%%%%%%%%%%%%%%%%%%%%%%%%%%%%%%%%%%%%%%%%%%%%%%%%%%%%%%%%%%%%%%%%
% Misc

\setbeamercolor{advantage text}{fg=safegreen!80!black}
\newenvironment{advantageenv}{\begin{altenv}%
    {\usebeamertemplate{advantage text begin}%
      \usebeamercolor[fg]{advantage text}%
     \usebeamerfont{advantage text}}
    {\usebeamertemplate{advantage text end}}{\color{.}}{}}{\end{altenv}}

\newcommand<>{\advantage}[1]{\hskip0pt\begin{advantageenv}#2#1\end{advantageenv}}
\newcommand<>{\Advantage}[1]{\advantage#2{\emph#2{#1}}}
\newcommand<>{\ADVANTAGE}[1]{\advantage#2{\textbf#2{#1}}}

\newcommand{\dn}[2][]{\tikz[baseline]\node[anchor=base,circle,inner sep=1pt,draw,every dn/.try,#1]{#2};}
\renewcommand{\iiad}{\ii[\color{safegreen}\textbf{+}]}  % advantage
\renewcommand{\iida}{\ii[\color{safered}\textbf{--}]} % disadvantage
\newcommand{\lecturetag}[3][]{%
  \tikz[baseline]\node[anchor=base,draw=black,fill=#2color,#1]{\color{black}Vorlesung #3};
}

\newcommand{\tikznode}[2][]{%
  \tikz[baseline,remember picture]\node[anchor=base,#1]{#2};%
}
\newenvironment<>{tikznodeenv}[1][]{%
  \begingroup%
  \tikzset{tikznode@style/.style={#1}}%
  \begin{lrbox}{\@tempboxa}%
}{\end{lrbox}%
  \tikz[remember picture]\node[inner sep=0,outer sep=0, tikznode@style]{\usebox{\@tempboxa}};
  \endgroup%
}

\newenvironment<>{visible}[1][]{\begin{scope}[visible on=#2,#1]}{\end{scope}}

\newsavebox{\@overlaybox}
\def\@overlayboxType{\vbox}
\def\overlayboxHBOX{\let\@overlayboxType=\hbox}
\newenvironment<>{overlaybox}[1][]{%
  % We set the overlay tikz node in the next shipout routine.
  % Thereby, we do not produce a \leavemode after the lrbox
  \only#2{%
    \AtBeginShipoutNext{%
      \AtBeginShipoutUpperLeftForeground{%
        \begin{tikzpicture}[remember picture, overlay]%
          \node[fill=white,draw] at (current page.center) [#1] {\usebox{\@overlaybox}};
        \end{tikzpicture}%
      }%
    }%
  }%
  % Produce the @conclusionoverlay
  \setbox\@overlaybox=\@overlayboxType\bgroup%
}{%
  \egroup%
  \global\setbox\@overlaybox=\copy\@overlaybox\relax%
}

%\usepackage{default}
\makeatletter
\newif\ifOnBeamerModeTransition
\newcommand{\slideselection}{1-}%
\newenvironment{handoutframeselect}[1][1-]{%
  \begingroup%
  \mode<handout>{%
    \gdef\beamer@currentmode{beamer}%
    \OnBeamerModeTransitiontrue%
    \renewcommand{\slideselection}{#1}}%
}{%
  \ifOnBeamerModeTransition%
    \OnBeamerModeTransitionfalse%
    \gdef\beamer@currentmode{handout}%
  \fi%
  \endgroup%
}
\makeatother


\makeatletter
\newsavebox{\@btModalBox}
\newenvironment<>{btModal}[1][]{%
  \begingroup%
  \gdef\@btModalStyle{fill=white,#1}%
  \def\@btModalOverlay{#2}% <overlayspec>
  \begin{lrbox}{\@btModalBox}%
  }{%
  \end{lrbox}%
  \global\setbox\@btModalBox=\copy\@btModalBox\relax%
  \expandafter\only\@btModalOverlay{%
    \AtBeginShipoutNext{%
      \AtBeginShipoutUpperLeftForeground{%
        \begin{tikzpicture}[remember picture, overlay]%
          \node[opacity=0.4,anchor=center,at=(current page.center),anchor=center,fill=luhgray,minimum width=\paperwidth,minimum height=\paperheight]{};
          \expandafter\node\expandafter[\@btModalStyle] at (current page.center) {\usebox{\@btModalBox}};
        \end{tikzpicture}%
      }%
    }%
  }%
  \endgroup%
}
\makeatother


\def\TAG#1{\hfill\raisebox{0.5\baselineskip}{\emph{#1}}}

  \newcommand{\InsertBibliography}{
    % if in \master mode, bib-items will be collected at the very end
    \ifx\master\undefined
      \subsection{Referenzen}
      \begin{frame}[allowframebreaks]
        \frametitle{Referenzen}
        \printbibliography
      \end{frame}
    \fi
  }
